% Complete documentation on the extended LaTeX markup used for Python
% documentation is available in ``Documenting Python'', which is part
% of the standard documentation for Python.  It may be found online
% at:
%
%     http://www.python.org/doc/current/doc/doc.html

\documentclass{manual}

\usepackage[pdftex]{graphics}

\title{Distributed Asynchronous Numerical Adaptive computing library}
\author{Nicolas Rougier}
\authoraddress{\email{Nicolas.Rougier@loria.fr}}
\date{August 13, 2007}
\release{1.0}
\makeindex
%\makemodindex


\begin{document}

\maketitle

% This makes the contents more accessible from the front page of the HTML.
\ifhtml
\chapter*{Front Matter\label{front}}
\fi
\input{copyright}

\begin{abstract}
\noindent
D.A.N.A is a python library that support distributed, asynchronous, numerical and
adaptive computation which is closely related to both the notion of artificial
neural networks and cellular automaton. However, there exist two major
differences. The first difference lies in the distributed nature of computation
that does not allow to implement "regular" neural networks which require non
local functions. The second difference lies in the asynchronous nature of
computations: units are evaluated in random order and updated immediately.
\end{abstract}

\tableofcontents

\chapter{Concept}

\begin{itemize}
\item [{\bf Excerpt from Dave Abrahams talk summary at BoostCon 07}\\
\\
]
{\em {\small "Python and C++ are in many ways as different as two languages could be:
     while C++ is usually compiled to machine-code, Python is interpreted.
     Python's dynamic type system is often cited as the foundation of its
     flexibility, while in C++ static typing is the cornerstone of its
     efficiency. C++ has an intricate and difficult compile-time meta-language,
     while in Python, practically everything happens at runtime.\\
     \\
     
     Yet for many programmers, these very differences mean that Python and C++
     complement one another perfectly. Performance bottlenecks in Python
     programs can be rewritten in C++ for maximal speed, and authors of
     powerful C++ libraries choose Python as a middleware language for its
     flexible system integration capabilities."}}
\end{itemize}

DANA is based on both C++ and Python. C++ ensures some decent speed for
simulation while python provides a powerful scripting language that can be used
for model design, interaction and visualization. The challenge is to export C++
objects to python with minimal effort and this is precisely what the boost
python library has been designed for. In the end, the user is able to build
various models simply by importing the relevant libraries into python.\\
\\

Hence, using DANA means writing some python scripts that import the core (and
possibly some other packages) in order to build and run a model. However, if
the core of DANA provides the user with a distributed numerical and
asynchronous computational paradigm, it does not provide any model at all. For
example, the core unit does not compute anything and this the responsability of
the user to write a unit class derived from the core unit that does compute
something useful.

\section{Overview}

\section{Implementation}



\chapter{Installation}
\section{Requirements}
\section{Installation on Linux}
\section{Installation on Mac OS X}
\section{Instalation on Windows}



\chapter{Core}

%\begin{center}
%    \includegraphics{images/core.png}
%\end{center}


\section{Unit}
\section{Layer}
\section{Map}
\section{Network}
\section{Environment}
\section{Model}

\appendix

\chapter{Revision history}


{\bf 2007-08-14 (1.0)}
\begin{itemize}
    \item * Initial publication
\end{itemize}



\chapter{About the book}

This document was generated using the LaTeX2HTML translator.\\

LaTeX2HTML is Copyright (c) 1993, 1994, 1995, 1996, 1997, Nikos Drakos,
Computer Based Learning Unit, University of Leeds, and Copyright (c) 1997, 1998,
Ross Moore, Mathematics Department, Macquarie University, Sydney.\\

The application of LaTeX2HTML to the Python documentation has been heavily
tailored by Fred L. Drake, Jr. Original navigation icons were contributed by
Christopher Petrilli.




\chapter{GNU Free Documentation License \label{License}}
\input{fdl}


%
%  The ugly "%begin{latexonly}" pseudo-environments are really just to
%  keep LaTeX2HTML quiet during the \renewcommand{} macros; they're
%  not really valuable.
%
%  If you don't want the Module Index, you can remove all of this up
%  until the second \input line.
%
%begin{latexonly}
%\renewcommand{\indexname}{Module Index}
%end{latexonly}
%\input{mod\jobname.ind}		% Module Index

%begin{latexonly}
%\renewcommand{\indexname}{Index}
%end{latexonly}
%\input{\jobname.ind}			% Index

\end{document}
